%
% firmware.tex
%
% Copyright The TTC 2.0 Contributors.
%
% TTC 2.0 Documentation
%
% This work is licensed under the Creative Commons Attribution-ShareAlike 4.0
% International License. To view a copy of this license,
% visit http://creativecommons.org/licenses/by-sa/4.0/.
%

%
% \brief Firmware project chapter.
%
% \author Gabriel Mariano Marcelino <gabriel.mm8@gmail.com>
%
% \version 0.3.0
%
% \date 2021/05/12
%


\chapter{Firmware} \label{ch:firmware}

This chapter describes the main characteristics of the firmware part of the TTC module.

\section{Product tree}

The product tree of the firmware part of the TTC 2.0 module is available in \autoref{fig:product-tree-fw}.

\begin{figure}[!ht]
    \begin{center}
        \includegraphics[width=\textwidth]{figures/product-tree-fw.pdf}
        \caption{Product tree of the firmware of the TTC 2.0 module.}
        \label{fig:product-tree-fw}
    \end{center}
\end{figure}

\section{Commands} \label{sec:commands}

The SPI\nomenclature{\textbf{SPI}}{\textit{Serial Peripheral Interface.}} commands of the TTC module are available in \autoref{tab:commands}. All commands are composed by an ID field (1 byte), the content of the command and a checksum at the end of the command (2 bytes). The used checksum algorithm is the CRC16-CCITT\nomenclature{\textbf{CRC}}{\textit{Cyclic Redundancy Check.}} \nomenclature{\textbf{CCITT}}{\textit{Comité Consultatif International Téléphonique et Télégraphique.}} (initial value = 0x0000, polynomial = 0x1021) the value is calculated with the entire packet (ID field + command content).

\begin{table}[!h]
    \centering
    \begin{tabular}{cll}
        \toprule[1.5pt]
        \textbf{ID} & \textbf{Name/Description} & \textbf{Content}\\
        \midrule
        0   & NOP                       & None \\
        1   & Read parameter/variable   & Parameter ID (1B) + Value (4B) + Checksum (2B) \\
        2   & Write parameter/variable  & Parameter ID (1B) + Value (4B) + Checksum (2B) \\
        3   & Transmit packet           & Packet data (1-220B) + Checksum (2B) \\
        4   & Receive packet            & Packet data (1-220B) + Checksum (2B) \\
        \bottomrule[1.5pt]
    \end{tabular}
    \caption{List of commands.}
    \label{tab:commands}
\end{table}

\section{Variables and Parameters} \label{sec:variables}

A list of all the variables of TTC with their identification number (ID) and variable type that can be read from the sensors and peripherals can be seen in \autoref{tab:ttc2-variables}.

\begin{longtable}[c]{cL{0.72\textwidth}lc}
    \toprule[1.5pt]
    \textbf{ID} & \textbf{Name/Description} & \textbf{Type} & \textbf{Access} \\
    \midrule
    0   & Device ID (0xCC2A or 0xCC2B)                                      & uint16 & R \\
    1   & Hardware version                                                  & uint8  & R \\
    2   & Firmware version (ex.: ``v1.2.3''' = 0x00010203)                  & uint32 & R \\
    3   & Time counter in millseconds                                       & uint32 & R \\
    4   & Reset counter                                                     & uint16 & R \\
    \multirow{18}{*}{5} & Last reset cause: & \multirow{18}{*}{uint8} & \multirow{18}{*}{R} \\
        & - 0x00 = No interrupt pending                                     &        &  \\
        & - 0x02 = Brownout (BOR)                                           &        &  \\
        & - 0x04 = RST/NMI (BOR)                                            &        &  \\
        & - 0x06 = PMMSWBOR (BOR)                                           &        &  \\
        & - 0x08 = Wakeup from LPMx.5 (BOR)                                 &        &  \\
        & - 0x0A = Security violation (BOR)                                 &        &  \\
        & - 0x0C = SVSL (POR)                                               &        &  \\
        & - 0x0E = SVSH (POR)                                               &        &  \\
        & - 0x10 = SVML\_OVP (POR)                                          &        &  \\
        & - 0x12 = SVMH\_OVP (POR)                                          &        &  \\
        & - 0x14 = PMMSWPOR (POR)                                           &        &  \\
        & - 0x16 = WDT time out (PUC)                                       &        &  \\
        & - 0x18 = WDT password violation (PUC)                             &        &  \\
        & - 0x1A = Flash password violation (PUC)                           &        &  \\
        & - 0x1C = Reserved                                                 &        &  \\
        & - 0x1E = PERF peripheral/configuration area fetch (PUC)           &        &  \\
        & - 0x20 = PMM password violation (PUC)                             &        &  \\
        & - 0x22 to 0x3E = Reserved                                         &        &  \\
    6   & Input voltage of the $\mu$C in mV                                 & uint16 & R \\
    7   & Input current of the $\mu$C in mA                                 & uint16 & R \\
    8   & Temperature of the $\mu$C in K                                    & uint16 & R \\
    9   & Input voltage of the radio in mV                                  & uint16 & R \\
    10  & Input current of the radio in mA                                  & uint16 & R \\
    11  & Temperature of the radio in K                                     & uint16 & R \\
    12  & Last valid command (uplink packet ID)                             & uint8  & R \\
    13  & RSSI of the last valid telecommand                                & uint16 & R \\
    14  & Temperature of the antenna module in K                            & uint16 & R \\
    \multirow{17}{*}{15} & Antenna module status bits:                      & \multirow{17}{*}{uint16} & \multirow{17}{*}{R} \\
        & - Bit 15: The antenna 1 is deployed (0) or not (1)                &        &   \\
        & - Bit 14: Cause of the latest activation stop for antenna 1       &        &   \\
        & - Bit 13: The antenna 1 deployment is active (1) or not (0)       &        &   \\
        & - Bit 11: The antenna 2 is deployed (0) or not (1)                &        &   \\
        & - Bit 10: Cause of the latest activation stop for antenna 2       &        &   \\
        & - Bit 9: The antenna 2 deployment is active (1) or not (0)        &        &   \\
        & - Bit 8: The antenna is ignoring the deployment switches (1) or not (0) &  &   \\
        & - Bit 7: The antenna 3 is deployed (0) or not (1)                 &        &   \\
        & - Bit 6: Cause of the latest activation stop for antenna 3        &        &   \\
        & - Bit 5: The antenna 3 deployment is active (1) or not (0)        &        &   \\
        & - Bit 4: The antenna system independent burn is active (1) or not (0) &    &   \\
        & - Bit 3: The antenna 4 is deployed (0) or not (1)                 &        &   \\
        & - Bit 2: Cause of the latest activation stop for antenna 4        &        &   \\
        & - Bit 1: The antenna 4 deployment is active (1) or not (0)        &        &   \\
        & - Bit 0: The antenna system is armed (1) or not (0)               &        &   \\
    16  & Antenna deployment status (0=never executed, 1=executed)          & uint8  & R \\
    17  & Antenna deployment hibernation (0=never executed, 1=executed)     & uint8  & R \\
    18  & TX enable (0=off, 1=on)                                           & uint8  & R/W \\
    19  & TX packet counter                                                 & uint32 & R \\
    20  & RX packet counter (valid packets)                                 & uint32 & R \\
    21  & TX packets available in the FIFO buffer                           & uint8  & R \\
    22  & RX packets available in the FIFO buffer                           & uint8  & R \\
    23  & Number of bytes of the first available packet in the RX buffer    & uint16 & R \\
    \bottomrule[1.5pt]
    \caption{Variables and parameters of the TTC 2.0.}
    \label{tab:ttc2-variables}
\end{longtable}

Each variable can be read or written using the commands ``Read Parameter'' and/or ``Write Parameter''.

\section{Development Layers}

The firmware flow of development goes from the low-level implementation (far right) with HA layer being register level operation to Tasks Layer with very abstract and high level code.

\subsection{HAL}

The HAL layer is the API Driverlib developed by Texas Instruments, it includes register manipulating functions to accelerate development. The TTC 2.0 uses HAL mostly to handle GPIO operations and serial communications such as SPI, UART and I$^2$C.

\subsection{Drivers}

Driver Layer is created to have the flexibility of the HAL layer but containing only the necessary abstraction in order to still be generic enough to support all the functionalities needed in Devices or other Drivers modules.

\subsection{Devices}

In this level of abstraction, the devices are used to create specific configurations with the Drivers Layer to build functions used in Tasks layer. In opposition to the Drivers layer the Devices don't communicate between themselves and are only used inside Tasks.

\subsection{RTOS}

The TTC 2.0 uses FreeRTOS kernel. Using an RTOS based kernel enables the system to maintain regularly all the routine functions such as housekeeping, sensors reading, check for receptions and to also deal with specific delays used in hardware, such as the radio. The level of priority of a task dictates who has the most preference of execution. Initial delay and period are used to determine the delay time to execute the task initialization after a boot, the regulates the time between the task executions. The stack is the amount of memory delimited for the execution of a task.

\subsection{System}

The System layer is used for housekeeping and management routines, it contains the system log, clocks setup and hooks. Its executions occurs in almost all the TTC 2.0 abstraction layers (expect the Tests layer) mostly for log purposes as seen in Figure \ref{fig:log_info}.

\begin{figure}[!h]
	\begin{center}
		\includegraphics[width=0.65\textwidth]{figures/ttc2-terminal-log.png}
		\caption{Example of log feedback received from TTC 2.0 during debug.}
		\label{fig:log_info}
	\end{center}
\end{figure}

\subsection{Tasks}

Tasks are the FreeRTOS threads equivalent and are the uppermost abstraction layer of code inside TTC 2.0 flow of execution. For each task is designated a level of priority, initial delay, period and stack size as shown in \autoref{tab:firmware-tasks}.

\begin{table}[!h]
    \centering

    \begin{tabular}{lccccc}
        \toprule[1.5pt]
        \textbf{Name}          & \textbf{Priority} & \textbf{Initial delay [ms]} & \textbf{Period [ms]} & \textbf{Stack [bytes]} \\
        \midrule
        Automatic Beacon       & High    & 60000 & 60000     & 300 \\
        Command Processing     & Highest & 0     & 100       & 500 \\
        Heartbeat              & Lowest  & 0     & 500       & 160 \\
        Housekeeping           & Medium  & 2000  & 10000     & 160 \\
        Radio Reset            & High    & 60000 & 60000     & 128 \\
        Startup                & Medium  & 0     & Aperiodic & 128 \\
        System Reset           & Medium  & 0     & 36000000  & 128 \\
        Time Control           & Medium  & 1000  & 1000      & 128 \\
        Uplink                 & Highest & 2000  & 500       & 500 \\
        Watchdog Reset         & Lowest  & 0     & 100       & 150 \\
        \bottomrule[1.5pt]
    \end{tabular}
    \caption{List of TTC 2.0 Tasks with configuration parameters.}
    \label{tab:firmware-tasks}
\end{table}

Each of the tasks presented in \autoref{tab:firmware-tasks} are described below:

\begin{itemize}
    \item \textbf{Automatic Beacon}: Automatically transmits a beacon packet if no transmission command is received within 60 seconds.
    \item \textbf{Command Processing}: Process incoming commands (physical interfaces).
    \item \textbf{Heartbeat}: Blinks a status LED at a rate of 1 Hz.
    \item \textbf{Housekeeping}: This task manages the general operation of the TTC.
    \item \textbf{Radio Reset}: Resets the radio at 600 seconds.
    \item \textbf{Startup}: Initializes the TTC 2.0 module.
    \item \textbf{System Reset}: Resets the microcontroller by software at every 10 hours.
    \item \textbf{Time Control}: Manages the system time.
    \item \textbf{Uplink}: Monitors radio for upcoming packages and stores it in memory.
    \item \textbf{Watchdog Reset}: Resets the watchdog timers at every 100 milliseconds.
\end{itemize}

\subsection{Libraries}

The Libraries are used for algorithm purposes and isn't related to any hardware. It's function is to remove redundancy of creating multiple identical structures for different driver modules.

\subsection{Tests}

Tests are used to verify and validate the functionality of the Drivers and Devices developed. There are three types of tests: the static test uses the MISRA C: 2012 \cite{misra} guidelines to check for C safety standards, the unitary tests uses the Cmocka library \cite{cmocka} with mockups of the hardware, to validate the module in an algorithmic level, and the last test verifies the integration between hardware and firmware.